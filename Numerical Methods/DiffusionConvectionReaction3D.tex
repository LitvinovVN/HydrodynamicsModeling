\documentclass[12pt]{article}

\usepackage[utf8]{inputenc}
\usepackage[english,russian]{babel}
\usepackage{amsmath}
\usepackage[unicode, pdftex]{hyperref}
\usepackage[usenames]{color}
\usepackage{colortbl}

\begin{document}

\section{Нестационарное уравнение диффузии-конвекции-реакции для трехмерной расчетной области}

\subsection{Постановка задачи}

Уравнение диффузии-конвекции-реакции:


\begin{equation}
	c_t' + uc_x' + vc_y' + wc_z' = (\mu c_x')_x' + (\mu c_y')_y' + (\nu c_z')_z' + f,	
\end{equation}

с граничными условиями:

\begin{equation}
	c_n'(x, y, z, t) = \alpha_n c + \beta_n,
\end{equation}

где
\textit{u, v, w} - составляющие вектора скорости,
\textit{f} - функция, описывающая интенсивность и распределение источников,
${\mu}$ - горизонтальная проекция коэффициента диффузионного (турбулентного) обмена,
${\nu}$ - вертикальная проекция коэффициента диффузионного (турбулентного) обмена.

\subsection{Построение дискретной модели}

Расчетная область вписана в прямоугольный параллелепипед. Для программной реализации математической модели транспорта веществ вводим равномерную расчетную сетку:

\begin{displaymath}
	\begin{split}
	w_h = \{t^n = n \tau, x_i = ih_x, y_j = jh_y, z_k = kh_z, n = \overline{0..N_x}, i = \overline{0..N_x},  \nonumber\\
	j = \overline{0..N_y}, k = \overline{0..N_z}, N_t \tau = l_x, N_yh_y = l_y, N_zh_z = l_z \}, \nonumber\\
	\end{split}
\end{displaymath}

где
${\tau}$ - шаг по временному направлению,
${h_x, h_y, h_z}$ - шаги по координатным осям пространства,
${N_t, N_x, N_y, N_z}$ - границы по времени и пространству.

Аппроксимация уравнения (1) по временной переменной выполняется на основе схем с весами. 

\begin{equation}
	\frac{\hat c - c}{\tau} + u\bar{c}_x' + v\bar{c}_y' + w\bar{c}_z' = (\mu\bar{c}_x')_x' + (\mu\bar{c}_y')_y' + (\mu\bar{c}_z')_z' + f ,		
\end{equation}

где  

$\bar{c} = \sigma\hat c + (1 - \sigma) с , \sigma \in [0,1] $ - вес схемы $(\sigma = 0,5; 0.75; 1)$

$c=c(x, y, z, t);$ $ $ $\hat c = (x, y, z, t + \tau)  $

\begin{center}
Рисунок 1
Разностный шаблон
\end{center}


\begin{center}
Рисунок 2
Параллилепипед с центром i, j, k
\end{center}

Ячейки представлены прямоугольными параллелипипедами, которые могут быть заполненными, пустыми или частично заполненными. 

Заполненность ячеек:
Центры ячеек и расчетные узлы сетки разнесены на $ \frac{h_x}{2}, \frac{h_y}{2}, \frac{h_z}{2}, $ по координатным направлениям $x, y, z$ соответственно.

Обозначим $O_{i,j,k}$ - степень заполненности объемной ячейки. 

\begin{center}
	Рисунок 3
	Вершины объемной ячейки
\end{center}

Получается, что окрестными ячейками узла ${i,j,k}$ являются 8 ячеек (см. рисунок 2).

Обозначим эти ячейки через координаты главных диагоналей (т. к. ячейки - это прямоугольные параллелепипеды).

Внизу:

1) ${(i-1,j+1,k-1)}$ - ${(i,j,k)}$ 

2) ${(i-1,j,k-1)}$ - ${(i,j-1,k)}$ 

3) ${(i,j,k-1)}$ - ${(i+1,j-1,k)}$ 

4) ${(i,j+1,k-1)}$ - ${(i+1,j,k)}$ 

Вверху:

1) ${(i-1,j+1,k)}$ - ${(i,j,k+1)}$ 

2) ${(i-1,j,k)}$ - ${(i,j-1,k+1)}$ 

3) ${(i,j,k)}$ - ${(i+1,j-1,k+1)}$ 

4) ${(i,j+1,k)}$ - ${(i+1,j,k+1)}$ 

{\color{red}{Читай метод конечных объемов (Рояк)}}


Для описания геометрии расчетного объема введем коэффициенты ${q_0, q_1,q_2,q_3,q_4,q_5,q_6}$ заполненности контрольных "объемов" ячейки ${(i,j,k)}$.

Значение ${q_0}$ характеризует степень заполненности объема ${V_0}$.

${q_0} - {V_0}: x\in (x_{i-\frac{1}{2} }, x_{i+\frac{1}{2} } ),  y\in (y_{j-\frac{1}{2} }, y_{j+\frac{1}{2} } ),  z\in (z_{k-\frac{1}{2} }, z_{k+\frac{1}{2} } )$

${q_6} - {V_1}: x\in (x_{i-\frac{1}{2} }, x_{i+\frac{1}{2} } ),  y\in (y_{j-\frac{1}{2} }, y_{j+\frac{1}{2} } ),  z\in (z_{k-\frac{1}{2} }, z_k )$

${q_5} - {V_2}: x\in (x_{i-\frac{1}{2} }, x_{i+\frac{1}{2} } ),  y\in (y_{j-\frac{1}{2} }, y_{j+\frac{1}{2} } ),  z\in (z_k, z_{k+\frac{1}{2} } )$

${q_2} - {V_3}: x\in (x_{i-\frac{1}{2} }, x_i ),  y\in (y_{j-\frac{1}{2} }, y_{j+\frac{1}{2} } ),  z\in (z_{k-\frac{1}{2} }, z_{k+\frac{1}{2} } )$

${q_1} - {V_4}: x\in (x_i, x_{i+\frac{1}{2} } ),  y\in (y_{j-\frac{1}{2} }, y_{j+\frac{1}{2} } ),  z\in (z_{k-\frac{1}{2} }, z_{k+\frac{1}{2} } )$

${q_4} - {V_5}: x\in (x_{i-\frac{1}{2} }, x_{i+\frac{1}{2} } ),  y\in (y_{j-\frac{1}{2} }, y_j ),  z\in (z_{k-\frac{1}{2} }, z_{k+\frac{1}{2} } )$

${q_3} - {V_6}: x\in (x_{i-\frac{1}{2} }, x_{i+\frac{1}{2} } ),  y\in (y_j, y_{j+\frac{1}{2} } ),  z\in (z_{k-\frac{1}{2} }, z_{k+\frac{1}{2} } )$

Будем называть $\Omega$ заполненные части объемов $V_m$, где $m=\overline{0...6}$. 
Таким образом, коэффициенты $g_m$ вычисляются по формулам:

$(q_0)_{i,j,k}=\frac{O_{i,j,k}+O_{i+1,j,k}+O_{i,j+1,k}+O_{i+1,j+1,k}+O_{i,j,k+1}+O_{i+1,j,k+1}+O_{i,j+1,k+1}+O_{i+1,j+1,k+1}}{8};$

$(q_6)_{i,j,k}=\frac{O_{i,j,k+1}+O_{i+1,j,k+1}+O_{i,j+1,k+1}+O_{i+1,j+1,k+1}}{4};$

$(q_5)_{i,j,k}=\frac{O_{i,j,k}+O_{i+1,j,k}+O_{i,j+1,k}+O_{i+1,j+1,k}}{4};$

$(q_2)_{i,j,k}=\frac{O_{i,j,k+1}+O_{i,j+1,k}+O_{i,j,k+1}+O_{i,j+1,k+1}}{4};$

$(q_1)_{i,j,k}=\frac{O_{i+1,j,k}+O_{i+1,j+1,k}+O_{i+1,j,k+1}+O_{i+1,j+1,k+1}}{4};$

$(q_4)_{i,j,k}=\frac{O_{i,j,k}+O_{i+1,j,k}+O_{i,j,k+1}+O_{i+1,j,k+1}}{4};$

$(q_3)_{i,j,k}=\frac{O_{i,j+1,k}+O_{i+1,j+1,k}+O_{i,j+1,k+1}+O_{i+1,j+1,k+1}}{4};$

Проинтегрируем по объему $\Omega_0$ уравнение (2), воспользуемся свойством линейности интеграла, в результате чего получим:

\begin{multline}
\iiint\limits_{\Omega_0} \frac{\hat c - c}{\tau}\,dxdydz + \iiint\limits_{\Omega_0} u\bar{c}_x'{\tau}\,dxdydz + \iiint\limits_{\Omega_0} v\bar{c}_y'{\tau}\,dxdydz + \iiint\limits_{\Omega_0} w\bar{c}_z'{\tau}\,dxdydz = \\
\iiint\limits_{\Omega_0} (\mu\bar{c}_x')_x'\,dxdydz + \iiint\limits_{\Omega_0} (\mu\bar{c}_y')_y'\,dxdydz + \iiint\limits_{\Omega_0} (\mu\bar{c}_z')_z'\,dxdydz + \iiint\limits_{\Omega_0} f\,dxdydz  
\end{multline}

Вычислим отдельно каждый из полученных интегралов.

\begin{equation}
	\iiint\limits_{\Omega_0} \frac{\hat c - c}{\tau}\,dxdydz \simeq (q_0)_{i,j,k}\iiint\limits_{V_0} \frac{\hat c - c}{\tau}\,dxdydz = (q_0)_{i,j,k}\frac{\hat c - c}{\tau}h_xh_yh_z
\end{equation}
Второй интеграл в формуле (4) принимает вид:

\begin{multline}
\iiint\limits_{\Omega_0} u\bar{c}_x'{\tau}\,dxdydz \simeq \iiint\limits_{\Omega_1} u\bar{c}_x'{\tau}\,dxdydz + \iiint\limits_{\Omega_2} u\bar{c}_x'{\tau}\,dxdydz \\ = (q_1)_{i,j,k}\iiint\limits_{V_1} u\bar{c}_x'{\tau}\,dxdydz + (q_2)_{i,j,k}\iiint\limits_{V_2} u\bar{c}_x'{\tau}\,dxdydz
\end{multline}






\end{document}
