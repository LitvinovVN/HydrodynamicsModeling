\documentclass[12pt]{article}

\usepackage[utf8]{inputenc}
\usepackage[english,russian]{babel}
\usepackage{amsmath}
\usepackage[unicode, pdftex]{hyperref}

\begin{document}

\section{Методы решения СЛАУ}

\subsection{Метод Гаусса}

Пусть исходная система выглядит следующим образом:

\begin{equation}
	\begin{cases}
		a_{11}x_1 + ... + a_{1n}x_n = b_1\\
		...\\
		a_{m1}x_1 + ... + a_{mn}x_n = b_m
	\end{cases}
\end{equation}

Её можно записать в следующем виде:

\begin{equation}
	\mathbf{Ax = b},
\end{equation}

где
\textbf{A} - основная матрица системы (коэффициенты левой части уравнения), \textbf{x} - вектор решений,
\textbf{b} - вектор свободных членов (правая часть уравнения)

\begin{equation}
	\mathbf{A} = 
	\begin{pmatrix}
		a_{11} & \dots & a_{1n} \\
		\vdots & \ddots & \vdots \\
		a_{m1} & \dots & a_{mn}
	\end{pmatrix}
	,
	\;\;
	\mathbf{x} = 
	\begin{pmatrix}
		x_1 \\
		\vdots \\
		x_n
	\end{pmatrix}
	\;\;
	\mathbf{b} = 
	\begin{pmatrix}
		b_1 \\
		\vdots \\
		b_n
	\end{pmatrix}
	.
\end{equation}

\href{https://zaochnik.com/spravochnik/matematika/issledovanie-slau/metod-gaussa/}{Метод Гаусса: основные понятия}

\paragraph{Алгоритм "Метод Гаусса".}

\paragraph{Реализация алгоритма на языке Python.}

\paragraph{Реализация алгоритма на языке C++.}

\paragraph{Реализация алгоритма на языке C\#.}

\subsection{Метод простой итерации}

\href{https://zaochnik.com/spravochnik/matematika/issledovanie-slau/iteratsionnye-metody-reshenija-slau/}{Метод простой итерации: основные понятия}

\section{Методы дискретизации}

\href{https://studfile.net/preview/5829857/}{Методы дискретизации для решения ОДУ}\\

Метод конечных разностей.\\

Левая разность.\\

Правая разность.\\

Центральная разность (только для ОДУ второго порядка).\\


\section{Методы решения задачи Коши}


Явный метод Эйлера.\\

Неявный метод Эйлера.

\end{document}